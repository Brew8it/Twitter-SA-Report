%%
%% This is file `kaumasterstemplate.tex',
%% generated with the docstrip utility.
%%
%% The original source files were:
%%
%% kauthesis.dtx  (with options: `masterstemplate')
%% 
%% This is a generated file.
%% 
%% Copyright (c) 2011-2014 Stefan Berthold <stefan.berthold@kau.se>
%% 
%% This file is part of the kauthesis bundle.
%% 
%% This work may be distributed and/or modified under the
%% conditions of the LaTeX Project Public License, either version 1.3
%% of this license or (at your option) any later version.
%% The latest version of this license is in
%%   http://www.latex-project.org/lppl.txt
%% and version 1.3 or later is part of all distributions of LaTeX
%% version 2005/12/01 or later.
%% 
%% This work has the LPPL maintenance status `author-maintained'.
%% 
%% The Current Maintainer and author of this work is Stefan Berthold.
%% 
%% This work consists of all files listed in manifest.txt.
%% 
%% kauthesis.dtx
%% Copyright (c) 2011-2015 Stefan Berthold <stefan.berthold@kau.se>
\documentclass{kaumasters} % available class options: garamond
\usepackage[swedish]{babel}
\usepackage[T1]{fontenc}
\usepackage[utf8]{inputenc}
\usepackage{titletoc}
\usepackage{geometry}

\renewcommand{\baselinestretch}{1.5} 


\makeatletter
\renewcommand\scriptsize{\@setfontsize\scriptsize{7}{8}}
\renewcommand\tiny{\@setfontsize\tiny{5}{6}}
\renewcommand\small{\@setfontsize\small{5}{6}}
\renewcommand\normalsize{\@setfontsize\normalsize{12}{12}}
\renewcommand\large{\@setfontsize\large{10.95}{15}}
\renewcommand\Large{\@setfontsize\Large{12}{16}}
\renewcommand\LARGE{\@setfontsize\LARGE{14.4}{18}}
\renewcommand\huge{\@setfontsize\huge{20.74}{30}}
\renewcommand\Huge{\@setfontsize\Huge{24}{36}}
\makeatother
\usepackage{sectsty}

\chapterfont{\Huge}
\sectionfont{\huge}
\subsectionfont{\LARGE}
\subsubsectionfont{\Large}
\paragraphfont{\Large}


\title{Twitter Sentiment Analysis}
\author{Johan Selberg, Johannes Bandgren}
\supervisor{Kerstin Andersson}
\examiner{Exam}
\institute{Department of Computer Science}
\place{Karlstad Universitet}
\begin{document}

\maketitle



\frontmatter
\begin{abstract}
  Abstract.
  \keywords keywords
\end{abstract}
\approvalpage%
\begin{acknowledgements}
  Thanks.
\end{acknowledgements}

\tableofcontents{}
\mainmatter

\newgeometry{top=20mm, bottom=20mm, right=22mm, left=22mm}

\chapter{Introduktion}
Internet of Things är ett begrepp för att beskriva ett samhälle där alla objekt, såväl som subjekt, är uppkopplade och unikt identifierbara på Internet. Mikrokontroller, som kan sesInternet of Things är ett begrepp för att beskriva ett samhälle där alla objekt, såväl som subjekt, är uppkopplade och unikt identifierbara på Internet. Mikrokontroller, som kan ses
Internet of Things är ett begrepp för att beskriva ett samhälle där alla objekt, såväl som subjekt, är uppkopplade och unikt identifierbara på Internet. Mikrokontroller, som kan ses
Internet of Things är ett begrepp för att beskriva ett samhälle där alla objekt, såväl som subjekt, är uppkopplade och unikt identifierbara på Internet. Mikrokontroller, som kan ses


\chapter{Bakgrund}
\section{Intro - syfte Varför vi gör denna studie}
\section{Förklara sentiment analys}
\subsection{Twitter Sentiment analys}
\section{Vilka modeller}
\section{Hur jämför vi modellerna? accuracy, F-Score, Precision}
\section{Summering}

\newpage

\chapter{Experiment}
\section{Intro}
\section{Feature selection}
\section{Modell 1}
\section{Modell N}
\section{Design}
\section{Implementation av modellerna}
\subsection{* ev GUI implementation om tid finns *}
\section{Summering}


\newpage

\chapter{Resultat}
\section{Intro}
\section{Resultatet mellan modellerna}
\subsection{Dataset 1 -> jämför resultat mellan modellerna}
\subsection{Dataset 2 -> jämför resultat mellan modellerna}
\subsection{Dataset 3 -> jämför resultat mellan modellerna}
\section{Implementations mässigt vilken modell är lättast?}
\section{implementations jämförelse (resultat VS förväntat)}
\section{Summering}

\newpage

\chapter{Slutsats}
\section{Sammanfattning}
\section{Problem}
\section{Begränsningar}
\section{Vidare utveckling}
\section{Slutord}

\restoregeometry%

\end{document}
\endinput
%%
%% End of file `kaumasterstemplate.tex'.
